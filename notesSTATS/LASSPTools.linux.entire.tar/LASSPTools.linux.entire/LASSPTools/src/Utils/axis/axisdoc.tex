\magnification=\magstep1
\parskip=3pt

\centerline{\bf Axis --- A program for making two dimensional graphs}
\bigskip
\beginsection 1. General information

Axis is a program for making two dimensional graphs.  It features
automatic scaling, logarithmic axes, error bars, and labels with Greek
letters, superscripts and subscripts, and special characters. The output
of axis is a file using a subset of the UNIX device independent graphics
commands, which can then be plotted on any device for which a plot
filter is available.  Axis supports most of the options of the UNIX
graph program, but also contains many additional features.

Axis was intended to be easy to use for simple plots.  It should be
possible to produce a simple plot using no options or only a few
options.  For a finished plot, with nice labels etc., many options may
be necessary.

In the simplest case, the input to axis is a series of lines, each line
containing a pair of values. For example:
$$\eqalign{
&1.0\quad   1.0\cr
&2.0\quad   2.1\cr
&3.0\quad   2.9\cr
}$$
With this input, axis will produce a graph with a line connecting all
the points.  The ranges of the x and y coordinates will be chosen
automatically to include all of the data.

There are a large number of options which modify the behavior of axis.
These options may be specified either in the command line of axis, or in
lines in the input beginning with the `\#' character. For example, to
set the x range to be 0 to 1.5, you can use
$$\hbox{ axis  -x  0  1.5  $<$  data  $>$  output}$$
The minus sign on the x is optional, but it is a convention for UNIX
arguments. You can also include a line in the data file:
$$\eqalign{
&\# x\quad 0\quad 1.5\cr
&0.0\quad  1.0\cr
&1.0\quad  2.0\cr
&2.0\quad  3.0\cr
}$$
(the points outside the range are ignored.) There may be any number of
option lines within the data file, and they may appear at any point.
For most options, such as the x or y ranges, it only makes sense to
specify the option once.  If you specify it more than once the last
specification wins.  Specifications inside the file win over
specifications in the command line. There are a number of options such
as line style, plotting symbol, or whether the data has error bars,
which can be changed in mid plot. Simply put an options line in the data
file at the appropriate point.

Axis can place a title on the graph, labels on the x and y axes, labels
at arbitrary points, or any character as a plotting symbol.  The
character set used in all of these labels is from the SLAC Unified
Graphics System.  It contains all the usual ASCII characters, Greek
Letters, Math symbols, and some special plotting characters.  The non
ASCII characters are specified by troff-like escape sequences. See
below.

Axis is a descendant of the UNIX graph command.  Many of the options are
compatible with graph, although axis does require that successive data
points be on separate lines.  The output of axis is in UNIX device
independent graphics format (see plot(5)). To display the results use
the plot filters (plot(1)).

The remainder of this document contains information on the various
options to axis.  They are divided into options affecting the data and
how it is plotted and options affecting the labeling.  The distinction
is not clear.  Appendix A displays the UGS character set, appendix B
summarizes the escape sequences for special characters, and appendix C
summarizes the options.

\beginsection 2.  Options affecting the data plotting

\beginsection 2.1 x and y ranges, logarithmic scales

An ``x" or a ``y" in the options string, followed by two numbers, forces the
x or y range to that given by the two numbers.  An ``x l" or a ``y l"
makes the corresponding axis logarithmic, and if it is followed
by two numbers they are the range.
An optional third number following the x or y is the distance
between major tick marks along the axis.
For example
$$\eqalign{
&x\quad 0\quad 2\cr
&x\ \ l\cr
&x\ \ l\quad 1\quad 1000\cr
}$$
are all legal options.

\beginsection 2.2 Automatic abscissas

If the ``-a" option is specified the x coordinate is missing from
the data.  Axis will then plot points evenly spaced in x.  If a single
number is after the ``-a", this number will be the spacing between
points, and if two numbers are present they will be the spacing
and the initial value of x.

\beginsection 2.3 Size and location of the graph.

Axis by default makes a square graph almost filling the page.
To change the size of the graph use ``-h number", where the number is
the fraction of the default height, and ``-w number" where the number
is the fraction of the default width.
To move the graph, use ``-u number" where the number is the fraction
of the graph size to move up, or ``-r number" to move right.
(It is usually necessary to move right a little if you want to
label the y axis. ``-r 0.1" usually works.)

\beginsection 2.4 Error bars

A ``e" (or a ``-e") in an options string means that subsequent lines of
input should contain three numbers: an x value, a y value, and an error
on the y value.  Axis will then plot error bars.  The ``e" option may
be changed in mid plot.  To turn off the
error bars and return to normal plotting, use an ``e0" in an options
line.

\beginsection 2.5 Linestyles


A ``m number" in the options changes the linestyle.  The recognized
values of the number are 0 for no lines (i.e. a point plot), 1 for
solid lines (this is the default), 2 for dotted lines, 3 for short
dashes, 4 for long dashes, and 5 for dot-dashed lines.  Not all output
devices will make all of these line styles.  This option may be 
changed in mid plot.

\beginsection 2.6 Labeling points and breaking lines

If the data in an input line is followed by a string in double quotes,
that string is used as a label for the point.  As noted before, the
character set contains a selection of plotting symbols which can be
used to plot the points.  To set a default label for each point
use the option `c ``label"', and to turn a default label off
use `c0'.  The label size can be adjusted by the option
``cs number".  The default size is 1.5, in units of a terminal's
standard character size.  See the section below on labels for a
instructions on the use of the special characters.

Any options line causes a break in the data.  That is, the point preceding
the options line will not be connected to the following point.
Thus an options line consisting only of a ``\#" will cause a break.
The default action of axis is also to break the graph after each labeled
point or error bar, that is, not to connect it to the next point.  To force
connections, use the option ``-lb0".
The option ``-lb1" restores the default action of breaking after labels
or error bars.  (``-b" is an archaic form of ``-lb0".)
Another way to introduce a break is to use a label consisting of a blank
(still in double quotation marks!).

\beginsection 2.7 Transposing x and y

The ``t" option transposes the x and y axes.  It doesn't work on
data with error bars.

\beginsection 2.8 Superimposing graphs

The ``-s" option prevents axis from putting a screen clear at the
beginning.
This allows superimposing graphs, or whencombined with the h,u,w, and
d options, allows more than one graph on the screen.

\beginsection 2.9 Grid options

Use the ``-g number" option to set the grid style. 0 means no grid,
1 means a frame with tick marks, 2 means a full grid, and 3 means
the bottom and left sides of the frame only.  The default
is 1.

Use ``-tsx number" and ``-tsy number" to set the tick mark style for the
x and y axes. 0 is no tick marks, 1 is big marks at the numbers and
little marks in between, and 2 is big marks at the numbers only.
The default is 1.

Use ``-nx" and/or ``-ny" to get no numbers on the x and/or y axes.

Use the ``-box" option to get a frame with no numbers or tick marks.

\beginsection 3. Labels

\def\bs{$\backslash$}
Axis will put titles on a graph, labels on the x and y axes, and labels
at arbitrary points.  There are a host of options to control this.  All
of the labels use the same character set.  Like all UNIX strings, they
must be enclosed in double quotes if they are more than one word.  To
get a real double quote into a label, use `\bs"'. Special characters are
given by troff like escape sequences, which are a backslash (``\bs")
followed by two characters.  For example, to get a Greek letter use
``\bs gX", which will produce the Greek equivalent of the Roman letter
X.  The case of the Greek letter will be the case of X.  (The g may be
either case, or you can use \bs*X for compatibility with troff.)

Labels may have superscripts and subscripts.  To start a superscript use
``\bs sp", and to end a superscript use ``\bs ep".  Use ``\bs sb" and
``\bs eb" to start and end a subscript.

To overstrike characters, underline characters, or fill a square root
sign, use ``\bs mk" (mark) and ``\bs rt" (return).  A ``mark"
remembers the current point, and a ``return" returns to it.  Thus to make
a square root sign with an ``a" in it use the square root symbol, the
overbar, and the a: ``\bs sq\bs mk\bs rn\bs
rta"

To get a real backslash into the label, use two backslashes (``\bs\bs
"). The escape sequences are tabulated in appendix B.

\beginsection 3.1 Specifying a title

Use the `lt ``title"' option to specify a title, where ``title" is
a character string enclosed in double quotes.  If a title is given,
there are other optional options (sorry about that) to modify it.
``lts number" modifies the title size, and the default size is 2.
``ltx number" and ``lty number" modify the placement of the title.
The default is top center of the graph, and the coordinate system used
is one in which the graph area runs from 0.0 to 1.0 in both the x and
y directions.

\beginsection 3.2 Labeling axes

The `lx ``string"' and `ly ``string"' options to put a label on the
x or y axis respectively.  The ``lxs number" and ``lys number" options
change the sizes of these labels.  The default axis label size is 1.5.
``lxx number",``lxy number",``lyx number", and ``lyy number" change the
x and y coordinates of the x and y axis labels.

\beginsection 3.3 Labels in the graph

As noted above, a line of the form
$$\hbox{ x\_value    y\_value    ``label" }$$
puts a label at the indicated coordinates (in the data's coordinate
system), and the ``-c label" option may be used to set a default label,
or plotting symbol, for each point.  The ``cs number" option changes
the label size.

\bigskip\bigskip
\line{\bf Appendix A: The UGS character set\hfil}
\bigskip
?
\bigskip\bigskip
\line{\bf Appendix B: escape sequences for special characters\hfil}
\bigskip
\halign{ #\hfil\quad&#\hfil\cr
Escape sequence& Result\cr
(backslash omitted!)&\cr
\noalign{\medskip}
\noalign{Plotting symbols}
pl&  plus sign\cr
cr&  cross\cr
di&  diamond\cr
sq&  square\cr
oc&  octagon\cr
fd&  fancy diamond\cr
fs&  fancy square\cr
fx&  fancy cross\cr
fp&  fancy plus\cr
bu&  burst\cr
\noalign{\medskip}
\noalign{Greek letters (both cases)}
ga,gA&  alpha\cr
gb,gB&  beta\cr
gg,gG&  gamma\cr
gd,gD&  delta\cr
ge,gE&  epsilon\cr
gz,gZ&  zeta\cr
gy,gY&  eta\cr
gh,gT&  theta\cr
gi,gI&  iota\cr
gk,gK&  kappa\cr
gl,gL&  lambda\cr
gm,gM&  mu\cr
gn,gN&  nu\cr
gc,gC&  xi\cr
go,gO&  omicron\cr
gp,gP&  pi\cr
gr,gR&  rho\cr
gs,gS&  sigma\cr
gt,gT&  tau\cr
gu,gU&  upsilon\cr
gf,gF&  phi\cr
gx,gX&  chi\cr
gq,gQ&  psi\cr
gw,gW&  omega\cr
\noalign{\medskip}
\noalign{Special symbols}
tm&  times\cr
di&  divide\cr
+-&  plus or minus\cr
$<$=&  less than or equal\cr
$>$=&  greater than or equal\cr
~=&  approximately equal\cr
n=&  not equal\cr
pt&  proportional to\cr
is&  integral sign\cr
li&  line integral\cr
pd&  partial derivative\cr
dl&  del (gradient)\cr
sr&  square root\cr
ul&  underline\cr
rn&  overbar (``run")\cr
hb&  h bar\cr
lb&  lambda bar\cr
de&  degree\cr
in&  infinity\cr
dg&  dagger\cr
dd&  double dagger\cr
$<$-&  left arrow\cr
-$>$&  right arrow\cr
da&  down arrow\cr
ua&  up arrow\cr
la&  left angle\cr
ra&  right angle\cr
ib&  interabang\cr
sc&  section\cr
ca&  cap (intersection)\cr
cu&  cup (union)\cr
mo&  member of\cr
nm&  not a member of\cr
ex&  exists\cr
al&  for all\cr
sb&  subset\cr
ds&  direct sum (xor)\cr
dp&  direct product\cr
sp&  superset\cr
\noalign{\medskip}
\noalign{Control characters}
sp&  start superscript\cr
ep&  end superscript\cr
sb&  start subscript\cr
eb&  end subscript\cr
mk&  mark location\cr
rt&  return to mark\cr
}
\break
\line{\bf Appendix C: summary of options\hfil}
\bigskip
Square brackets indicate optional modifiers.  The minus sign is
unnecessary, just conventional.
\medskip
\halign{ #\hfil\quad&#\hfil\cr
-x [l] [xmin xmax] [xquant]&  range of x, and logarithmic flag\cr
-y [l] [xmin xmax] [yquant]&  range of y, and logarithmic flag\cr
-a [spacing] [initial]&  automatic abscissas\cr
-m style&  linestyle\cr
-g style&  grid style\cr
-tsx style& x tick mark style\cr
-tsy style& y tick mark style\cr
-nx& don't number x axis\cr
-ny& don't number y axis\cr
-box& don't number or tick axes\cr
-s&  don't erase\cr
-e&  data with errors\cr
-e0&  end data with errors\cr
-c ``string"&  default plot symbol\cr
-c0&  turn off default symbol\cr
-cs number&  plot symbol size\cr
-lb1&  break after labels\cr
-lb0&  don't break at labels\cr
-h number&  fraction of height to use\cr
-w number&  fraction of width to use\cr
-u number&  move up\cr
-r number&  move right\cr
-lt ``string"&  title\cr
-lx ``string"&  x axis label\cr
-ly ``string"&  y axis label\cr
-lts number&  title size\cr
-lxs number&  x label size\cr
-lys number&  y label size\cr
-ltx number&  title x coordinate\cr
-lty number&  title y coordinate\cr
-lxx number&  x label x coordinate\cr
-lxy number&  x label y coordinate\cr
-lyx number&  y label x coordinate\cr
-lyy number&  y label y coordinate\cr
}
\bye
